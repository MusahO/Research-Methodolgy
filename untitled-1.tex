\documentclass{article}


\usepackage[a4paper,width=150mm,top=25mm,bottom=25mm]{geometry}
\usepackage{fancyhdr}
\pagestyle{fancy}
\fancyhf{}
\fancyfoot[R]{\thepage}
\renewcommand{\headrulewidth}{0pt}
 \renewcommand{\footrulewidth}{0pt}


\usepackage[export]{adjustbox}

\begin{document}

\begin{figure}[h]
  \centerline{\small MAKERERE 
  \includegraphics[width=0.2\textwidth]  {mak-logo-sm.png} UNIVERSITY\\}
 \end{figure}

\centerline{COLLEGE OF COMPUTING AND INFORMATION SCIENCES\\}
\centerline{DEPARTMENT OF COMPUTER SCIENCE\\}
\centerline{COURSEWORK: RESEARCH METHODOLOGY(BIT 2207)\\}
\centerline{LECTURER: ERNERST MWEBAZE\\}


\begin{titlepage}

	\begin{center}
	\line(1,0){300}\\

	\huge{\bfseries Machine learning for  crime prevention in city suburbs}\\
	[2mm]
	\line(1,0){200}\\
	[1.5cm]

	\end{center}

	\begin{flushright}
	
	\textsc{\large Odeke Moses \\}
	\# 216016894 \\
	

	\end{flushright}
\end{titlepage}

\tableofcontents
\cleardoublepage

\section{INTRODUCTION}\label{sec:intro}

Crime is defined as “An action or an instance of negligence that is injurious to the public welfare or morals or to the interests of the state and that is legally prohibited”


 In the sociological field, crime is the breach of a rule or law for which some which some governing authority or force may ultimately prescribe a punishment. The word crime is derived from the Latin word Crimen


\section{TYPES OF CRIME}

\subsection {Crime against person}
against person / people includes murder, attempt to murder, hurt, rioting, and assault on public servant, rape or sexual assault, kidnapping or abduction etc. 
 
\subsection {Crime against property}
Crime against property includes highway dacoit, bank dacoit, petrol pump dacoit, other dacoit, highway robbery, bank robbery, petrol pump robbery, other robbery, burglary, cattle theft, motor vehicle, other theft etc. 
 
\subsection {White collar crimes} 
It is defined as “a crime committed by a person of respectability and high social status in the course 
of his occupation”. 
  White collar crime includes bank fraud, black mail, cellular phone fraud, computer fraud, credit card fraud, insider trading, insurance fraud, theft, and tax evasion etc. 

\subsection {Victimless crimes}
Victimless crimes include drug addiction, prostitute, and suicide etc. 


\section{BACKGROUND} 
I have chosen this topic because of increasing crime rate in the city suburbs during the last decade. We have to study the causes and factors behind this critical issue and also give some suggestions and recommendations to overcome this problem.

\section{OBJECTIVES}
1)	To use big data and machine learning to try when and where crime will take place by analyzing existing data on past crimes.


\section{LIMITATIONS}
As I am a student, I don’t have sufficient amount of money to fulfill this task accurately and precisely. Being a student, I have to face the time managing problem; I have no sufficient time due to other subjects and practical work to be done.

\section{Description of Reasearch}
Applied Research\\
Quantitavie Research\\
Analytical Research

\end{document}
