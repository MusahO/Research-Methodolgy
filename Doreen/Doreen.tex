\documentclass[10pt,a4paper]{article}
\usepackage[utf8]{inputenc}
\usepackage{amsmath}
\usepackage{amsfonts}
\usepackage{amssymb}
\author{Nassimbwa Doreen Patricia}
\begin{document}
\begin{flushright}
NASSIMBWA DOREEN \newline
216004538 \newline
16/U/110016/EVE \newline
\end{flushright}
\begin{center}
BIT 2207 RESEARCH METHODOLOGY \newline
\textbf {Title}\newline 
\textbf {CAUSES OF INCREASING POPULATION IN UGANDA} \newline
\textbf {problem statement}\newline 
\end{center}
\begin{flushleft}\textbf{Decline in the Death Rate:} At the root of overpopulation is the difference between the overall birth rate and death rate in population. If the number of children born each year equals to the number of adults that die, then the population will stabilize. Talking about overpopulation shows that while there are many factors that can increase the death rate for short periods of time, the ones that increase the birth rate do so over a long period of time.\newline
\end{flushleft}
\begin{flushleft}
\textbf{Technological Advancement in Fertility Treatment:} With latest technological advancement and more discoveries in medical science, it has become possible for couples who are unable conceive to undergo fertility treatment methods and have their own babies.\newline
\end{flushleft}
solutions\newline \hfill
\begin{flushleft}
textbf{Better Education:} one of the measures is to implement policies reflecting social change. Educating the masses helps them understand the need to have one or two children at the most. Similarly, education plays a vital role in understanding latest technologies such as CloudDesktopOnline that are making huge waves in the world of computing.\newline
\end{flushleft}
\textbf{Objectives:}\newline
\underline{Main}\newline
The higher the population, the bigger the market leading to a rapid development of the area.\newline
\underline{Other Objectives}\newline
The increasing population leads to the increase in taxation.\newline
There is creation of job employment.\newline
\textbf{Introduction}\newline
\begin{flushleft}
Economic history is the study of people and their way of life, it is concerned with all the people, not simply the rulers or decision makers, and it is therefore very important that economic historians  should know about the number of people in the community they are studying; in other words they must concern themselves with the size of the population. As historians they also need to know how the size of the population has changed over time; has it increased in size, or decreased. and even more important, why and how have the increase or decrease occurred? This chapter will attempt to answer these questions. Over the last one hundred years  or so the size of the populations of Kenya and Uganda has been of more than academic interest to students of history. Since 1948 demographers have shown that the population in both countries has grown at the rate of between 2.9 percent and 3.5 percent per year. As we shall illustrate in this chapter, this rate of growth is at least as high as that of any other known population growth rate at any time in history, anyway the world. Such a rate is unprecedented and we need to try to discover the factors which have led to this situation.\newline
\end{flushleft}
\end{document}